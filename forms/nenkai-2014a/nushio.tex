% DON'T CHANGE THE NEXT LINE! (次の行は変更しないで下さい)
\documentclass{2014a}
%%%%%%%%%%%%%%%%%%%%%%%%%%%%%%%%%%%%%%%%%%%%%%%%%%%%%%%%%%%%%%%%
% 講演者についての情報
\SpeakerInfo
%%%%%%%%%%%%%%%%%%%%%%%%%%%%%%%%
% 総講演数(半角数字)
{1}
%%%%%%%%%%%%%%%%%%%%%%%%%%%%%%%%
% 氏名
{村主崇行}
%%%%%%%%%%%%%%%%%%%%%%%%%%%%%%%%
% 氏(ひらがな)
{むらぬし}
%%%%%%%%%%%%%%%%%%%%%%%%%%%%%%%%
% 名(ひらがな)
{たかゆき}
%%%%%%%%%%%%%%%%%%%%%%%%%%%%%%%%
% 所属機関(例:◯◯大学、◯◯研究所)
{日本天文学会}
%%%%%%%%%%%%%%%%%%%%%%%%%%%%%%%%
% 会員種別(半角英小文字)
%   a=正会員(一般)
%   b=正会員(学生)
%   c=準会員(一般)
%   d=準会員(学生)
%   e=非会員(一般)
%   f=非会員(学生)
{a}
%%%%%%%%%%%%%%%%%%%%%%%%%%%%%%%%
% 会員番号(半角数字4桁、非会員は不要)
{5044}
%%%%%%%%%%%%%%%%%%%%%%%%%%%%%%%%
% 電話番号(半角数字とハイフン)
{075-753-7080}
%%%%%%%%%%%%%%%%%%%%%%%%%%%%%%%%
% ファックス番号(半角数字とハイフン)
{075-753-7020}
%%%%%%%%%%%%%%%%%%%%%%%%%%%%%%%%
% メールアドレス(半角文字)
{muranushi.takayuki.3r@kyoto-u.ac.jp}
%%%%%%%%%%%%%%%%%%%%%%%%%%%%%%%%
% 住所
{〒606-8502 京都市左京区北白川追分町 京都大学基礎物理学研究所}
%%%%%%%%%%%%%%%%%%%%%%%%%%%%%%%%%%%%%%%%%%%%%%%%%%%%%%%%%%%%%%%%
% 講演についての情報
\PaperInfo
%%%%%%%%%%%%%%%%%%%%%%%%%%%%%%%%
% 講演分野(半角英大文字+半角数字)
%   企画セッションは世話人が記入します。
%   J1=高密度星(BH・NS)
%   J2=高密度星(WD・GRB・その他)
%   K=超新星爆発
%   L=太陽系
%   M=太陽
%   N=恒星
%   P1=星・惑星形成(星形成)
%   P2=星・惑星形成(系外惑星)
%   Q=星間現象
%   R=銀河
%   S=活動銀河核
%   T=銀河団
%   U=宇宙論
%   V1=地上観測機器(電波)
%   V2=地上観測機器(その他)
%   W1=飛翔体観測機器(X線・γ線)
%   W2=飛翔体観測機器(その他)
%   X=銀河形成
%   Y=天文教育・その他
{M}
%%%%%%%%%%%%%%%%%%%%%%%%%%%%%%%%
% 講演形式(半角英小文字)
%   a=口頭講演
%   b=ポスター講演(口頭有)
%   c=ポスター講演(口頭無)
{b}
%%%%%%%%%%%%%%%%%%%%%%%%%%%%%%%%
% キーワード(複数ある場合は「、」で区切)
%   観測機器分野(地上・飛翔体)では関連の深いプロジェクト名・
%   衛星計画名などがあればご記入ください。
{宇宙天気、太陽フレア}
%%%%%%%%%%%%%%%%%%%%%%%%%%%%%%%%
% 題名
%   改行等を指示するコマンドは入れないで下さい。
{ビッグデータ分析手法を用いた宇宙天気予報アルゴリズムの詳細}
%%%%%%%%%%%%%%%%%%%%%%%%%%%%%%%%
% 氏名(所属)(複数の場合は「, 」で区切)
%   改行等を指示するコマンドは入れないで下さい。
{村主崇行,羽田裕子,磯部洋明,柴田一成,柴山拓也(京都大学),
根本茂(京都大学・株式会社ブロードバンドタワー),
駒崎健二(株式会社ブロードバンドタワー)}
%%%%%%%%%%%%%%%%%%%%%%%%%%%%%%%%%%%%%%%%%%%%%%%%%%%%%%%%%%%%%%%%
\begin{document}
%%%%%%%%%%%%%%%%%%%%%%%%%%%%%%%%%%%%%%%%%%%%%%%%%%%%%%%%%%%%%%%%
% 本文開始
%%%%%%%%%%%%%%%%%%%%%%%%%%%%%%%%%%%%%%%%%%%%%%%%%%%%%%%%%%%%%%%%

太陽フレアの発生とその影響を予測・予報することは、
太陽物理学の大きな目標の一つである。
これまで、フレア予測や宇宙天気予報の研究は多くなされているものの、黒点
やActive Regionの抽出・同定などに人手の介入を必要としていた。そこで我々
は近年飛躍的に増大している観測データを余すところなく利用できるよう、完
全に自動化された宇宙天気予報アルゴリズムの開発を目指している。

フレア活動の機械的に得られる指標としてGOES観測衛星による全球X線Fluxを、
太陽の状態を得るための入力データとしてSDO/HMIの全球磁場画像を選び、
西暦2011-2012年の二年間のデータを元に、
24時間将来までの全球X線Fluxの最大値を予報するアルゴリズムの作成を
試みている。

以下に、具体的な手順を述べる。
各時点での太陽画像を1024x1024の解像度に縮小した上でウェーブレット変換を
施す。ウェーブレット空間において、波長が共通の部分空間ごとに場の値を
積分、あるいは二乗積分したものを特徴量とする。こうして各時点で
150個ほどの特徴量が得られる。各時点において、これら特徴量を入力、「その時点から24時間将来までの全球X線Fluxの最大値」を出力とする標本データを作る。
標本データを訓練用と試験用に分割し、
訓練データをよくフィットする関数をsupport vector regressionにより作成する。
得られた近似関数を試験データに適用して予報精度を評価する。

本発表では、我々の予報手法の技術的詳細を展示し、予測精度の現状をHSS(Heidke Skill Score)およびTSS(True skill statistic)を用いて既存の予報研究と比較する。


%%%%%%%%%%%%%%%%%%%%%%%%%%%%%%%%%%%%%%%%%%%%%%%%%%%%%%%%%%%%%%%%
% 本文終了
%%%%%%%%%%%%%%%%%%%%%%%%%%%%%%%%%%%%%%%%%%%%%%%%%%%%%%%%%%%%%%%%
\end{document}
